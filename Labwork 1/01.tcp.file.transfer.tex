\documentclass[a4paper,12pt]{article}
\usepackage[utf8]{inputenc}
\usepackage{geometry}
\usepackage{graphicx}
\usepackage{listings}
\usepackage{xcolor}
\usepackage{float}
\usepackage{hyperref}


\geometry{margin=1in}


\definecolor{codegreen}{rgb}{0,0.6,0}
\definecolor{codegray}{rgb}{0.5,0.5,0.5}
\definecolor{codepurple}{rgb}{0.58,0,0.82}
\definecolor{backcolour}{rgb}{0.95,0.95,0.92}

\lstset{
    language=C,
    backgroundcolor=\color{backcolour},   
    commentstyle=\color{codegreen},
    keywordstyle=\color{magenta},
    numberstyle=\tiny\color{codegray},
    stringstyle=\color{codepurple},
    basicstyle=\ttfamily\footnotesize,
    breakatwhitespace=false,         
    breaklines=true,                 
    captionpos=b,                    
    keepspaces=true,                 
    numbers=left,                    
    numbersep=5pt,                  
    showspaces=false,                
    showstringspaces=false,
    showtabs=false,                  
    tabsize=2
}


\title{\textbf{Distributed Systems}\\Practical Work 1: TCP File Transfer}
\author{
    \textbf \\    Student name: Nguyen Minh Cuong - 23BI14467 \\
}
\date{\today}

\begin{document}

\maketitle

\section{Introduction}
The objective of this lab work is to implement a file transfer system using the TCP/IP protocol over the CLI (Command Line Interface). The system consists of a Server that listens for connections and a Client that initiates the transfer. We implemented the solution using the \textbf{C language} and standard POSIX Sockets on \textbf{Kali Linux}.

\section{Protocol Design}
Since TCP is a stream-oriented protocol, the receiver does not know where a file ends. Therefore, we designed a simple application-layer protocol to ensure data integrity.

\subsection{The Protocol Steps}
\begin{enumerate}
    \item \textbf{Handshake (Metadata):} The Client sends a formatted string containing the filename and file size.
    \begin{itemize}
        \item Format: \texttt{filename|filesize} (e.g., \texttt{mytest.txt|1024})
    \end{itemize}
    \item \textbf{Acknowledgment:} The Server parses the metadata, opens a file for writing, and sends back an \texttt{"OK"} signal.
    \item \textbf{Data Transmission:} The Client reads the file in chunks (Buffer size: 1024 bytes) and sends raw binary data.
    \item \textbf{Termination:} The Server counts the received bytes. When \texttt{total\_received == filesize}, it closes the connection.
\end{enumerate}

\subsection{Interaction Diagram}
\begin{figure}[H]
\centering
\begin{verbatim}
    CLIENT                                      SERVER
      |                                            |
      | --- 1. Metadata: "file.txt|500" ---------> |
      |                                            |
      | <----------- 2. Signal: "OK" ------------- |
      |                                            |
      | --- 3. Data Chunk (1024 bytes) ----------> |
      | --- 3. Data Chunk (remaining) -----------> |
      |                                            |
      |           (Server counts bytes)            |
      | <----------- 4. Close connection --------- |
\end{verbatim}
\caption{File Transfer Protocol Design}
\end{figure}

\section{System Organization}
The architecture follows the standard TCP State Diagram.

\begin{itemize}
    \item \textbf{Server:} 
    \begin{enumerate}
        \item \texttt{socket()}: Create endpoint.
        \item \texttt{setsockopt()}: Enable \texttt{SO\_REUSEADDR} to prevent "Address already in use" errors.
        \item \texttt{bind()}: Attach to port 8080.
        \item \texttt{listen()}: Wait for client.
        \item \texttt{accept()}: Handshake and create a new file descriptor for the session.
    \end{enumerate}
    
    \item \textbf{Client:}
    \begin{enumerate}
        \item \texttt{socket()}: Create endpoint.
        \item \texttt{connect()}: Connect to Server IP (127.0.0.1) on port 8080.
        \item \texttt{send()/recv()}: Transfer data.
    \end{enumerate}
\end{itemize}

\section{Implementation Details}

\subsection{Server-side (Handling Data)}
The server extracts the file size and loops until all bytes are received.

\begin{lstlisting}[caption=Server Code Snippet]
// Parse metadata
sscanf(buffer, "%[^|]|%ld", filename, &filesize);
send(sockfd, "OK", 2, 0); // Ack

// Receive loop
while (total_received < filesize) {
    bzero(buffer, BUFFER_SIZE);
    int n = recv(sockfd, buffer, BUFFER_SIZE, 0);
    if (n <= 0) break;
    
    fwrite(buffer, 1, n, fp);
    total_received += n;
}
\end{lstlisting}

\subsection{Client-side (Sending Data)}
The client waits for the "OK" signal before sending the actual file content.

\begin{lstlisting}[caption=Client Code Snippet]
// Send metadata
sprintf(buffer, "%s|%ld", filename, filesize);
send(sock, buffer, sizeof(buffer), 0);

// Wait for Server ACK
recv(sock, buffer, BUFFER_SIZE, 0); 

// Send file content
while(1) {
    int bytes_read = fread(buffer, 1, BUFFER_SIZE, fp);
    if (bytes_read <= 0) break;
    send(sock, buffer, bytes_read, 0);
}
\end{lstlisting}

\section{Results}
I successfully tested the system on Kali Linux. The client transferred \texttt{mytest.txt} and the server received it as \texttt{recv\_mytest.txt} with identical content.


 \begin{figure}[H]
     \centering
     \includegraphics[width=0.9\textwidth]{result.png}
     \caption{Successful file transfer on Kali Linux Terminal}
 \end{figure}

\section{Conclusion}
Through this practical work, we have successfully implemented a basic File Transfer system using the Client-Server model and TCP/IP sockets in C.

\begin{itemize}
    \item \textbf{Protocol Design:} We designed a simple application-layer protocol that sends metadata (filename and size) before the actual data, ensuring the server knows exactly how much data to read.
    \item \textbf{Socket Implementation:} We mastered the usage of POSIX socket primitives including \texttt{socket}, \texttt{bind}, \texttt{listen}, \texttt{accept}, and \texttt{connect}.
    \item \textbf{Error Handling:} We successfully handled common network errors, specifically the "Address already in use" error on the server side by implementing the \texttt{SO\_REUSEADDR} socket option.
\end{itemize}



\end{document}